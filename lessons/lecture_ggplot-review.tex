\PassOptionsToPackage{unicode=true}{hyperref} % options for packages loaded elsewhere
\PassOptionsToPackage{hyphens}{url}
%
\documentclass[]{article}
\usepackage{lmodern}
\usepackage{amssymb,amsmath}
\usepackage{ifxetex,ifluatex}
\usepackage{fixltx2e} % provides \textsubscript
\ifnum 0\ifxetex 1\fi\ifluatex 1\fi=0 % if pdftex
  \usepackage[T1]{fontenc}
  \usepackage[utf8]{inputenc}
  \usepackage{textcomp} % provides euro and other symbols
\else % if luatex or xelatex
  \usepackage{unicode-math}
  \defaultfontfeatures{Ligatures=TeX,Scale=MatchLowercase}
\fi
% use upquote if available, for straight quotes in verbatim environments
\IfFileExists{upquote.sty}{\usepackage{upquote}}{}
% use microtype if available
\IfFileExists{microtype.sty}{%
\usepackage[]{microtype}
\UseMicrotypeSet[protrusion]{basicmath} % disable protrusion for tt fonts
}{}
\IfFileExists{parskip.sty}{%
\usepackage{parskip}
}{% else
\setlength{\parindent}{0pt}
\setlength{\parskip}{6pt plus 2pt minus 1pt}
}
\usepackage{hyperref}
\hypersetup{
            pdftitle={ggplot2 review},
            pdfborder={0 0 0},
            breaklinks=true}
\urlstyle{same}  % don't use monospace font for urls
\setlength{\emergencystretch}{3em}  % prevent overfull lines
\providecommand{\tightlist}{%
  \setlength{\itemsep}{0pt}\setlength{\parskip}{0pt}}
\setcounter{secnumdepth}{0}
% Redefines (sub)paragraphs to behave more like sections
\ifx\paragraph\undefined\else
\let\oldparagraph\paragraph
\renewcommand{\paragraph}[1]{\oldparagraph{#1}\mbox{}}
\fi
\ifx\subparagraph\undefined\else
\let\oldsubparagraph\subparagraph
\renewcommand{\subparagraph}[1]{\oldsubparagraph{#1}\mbox{}}
\fi

% set default figure placement to htbp
\makeatletter
\def\fps@figure{htbp}
\makeatother


\title{ggplot2 review}
\date{}

\begin{document}
\maketitle

\hypertarget{repaso-ggplot2}{%
\section{Repaso ggplot2}\label{repaso-ggplot2}}

\hypertarget{ggplot2-figuras-combinando-capas}{%
\subsection{ggplot2: figuras combinando
capas}\label{ggplot2-figuras-combinando-capas}}

ggplot2 asocia distintas funciones a \emph{capas} que representan
distintas partes de un gráfico. En su forma más sencilla tenemos dos
funciones:

\begin{verbatim}
ggplot(data = DATA) + GEOM_FUNCTION(mapping = aes(MAPPINGS))
\end{verbatim}

En donde \texttt{DATA} es el dataframe que contiene las columnas a
graficar, \texttt{GEOM\_FUNCTION} representa una función para asignar
una geometría (por ejemplo, \texttt{geom\_point} o \texttt{geom\_line})
y \texttt{MAPPINGS} representa las variables que quiero dibujar (x e y,
o tal vez solo x).

En este ejemplo general y sencillo tengo cuatro elementos: el lienzo
(vacío), los datos, las variables y la geometría.

\hypertarget{ejemplito}{%
\subsection{Ejemplito}\label{ejemplito}}

May the source be with you

\begin{verbatim}
library(dplyr)
data(starwars)
glimpse(starwars) # glimpse es una versión más simple de str()

# un primer plot:
ggplot(data = starwars) + geom_point(mapping = aes(height, mass))

# sin nombrar los argumentos explícitamente:
ggplot(starwars) + geom_point(aes(height, mass))

# vale quebrarlo en dos líneas, si dejo el '+' al final de la línea
ggplot(starwars) + 
geom_point(aes(height, mass))
\end{verbatim}

\hypertarget{mappings}{%
\subsection{Mappings}\label{mappings}}

El llamado \emph{mapping} es uno de los argumentos de las funciones de
geometría y el objeto que hay que pasarle se genera con \texttt{aes()}.
El mapping conecta elementos de los datos con información visual en el
gráfico.

Por ejemplo, para asignar variables a ejes, en \texttt{aes()}
asignaremos una columna de nuestro daraframe a los argumentos \texttt{x}
y (opcionalmente) \texttt{y}: \texttt{aes(x\ =\ height,\ y\ =\ mass)}.
Así conectamos posición en el gráfico con columnas del dataframe.

Podemos manipular otros aspectos estéticos del gráfico con
\texttt{aes()}:

\begin{verbatim}
ggplot(starwars) + geom_point(aes(height, mass))
# o, mapeando elementos estéticos a variables:
ggplot(starwars) + geom_point(aes(height, mass, color=gender))
# podemos bypasear el mapping y definir un color único para la capa entera.
ggplot(starwars) + geom_point(aes(height, mass), color="blue")

ggplot(starwars) + geom_point(aes(height, mass, size=gender))
ggplot(starwars) + geom_point(aes(height, mass, shape=gender))
ggplot(starwars) + geom_line(aes(height, mass, linetype=gender)) 
ggplot(starwars) + geom_point(aes(height, mass, color=gender, shape=eye_color))
\end{verbatim}

\hypertarget{jerarquuxedas}{%
\subsection{Jerarquías}\label{jerarquuxedas}}

Las funciones de geometría pueden llevar su propio mapping, o pueden
heredar un mapping de la función \texttt{ggplot}.

\begin{verbatim}
# las siguentes dos expresiones funcionan igual
ggplot(starwars, aes(height, mass)) + geom_point()
ggplot(starwars) + geom_point(aes(height, mass))
\end{verbatim}

Esto nos da flexibilidad, porque podemos usar distintas variables en
distintas capas:

\begin{verbatim}
ggplot(starwars) + 
geom_point(aes(height, mass)) + 
geom_smooth(aes(height, birth_year))
\end{verbatim}

Podemos inclusive combinar data.frames distintos (no olvidar poner `data
='):

\begin{verbatim}
ggplot() + 
geom_point(data = starwars, aes(height, mass), color = "red") + 
geom_point(data = morley, aes(Run, Speed), color = "blue")
\end{verbatim}

\hypertarget{texto}{%
\subsection{Texto}\label{texto}}

Podemos poner texto de diversas maneras, por ejemplo:

\begin{verbatim}
ggplot(starwars) + 
    geom_point(aes(height, mass)) + 
    labs(colour = "gender", title = "Guerra de las Galaxias", subtitle = "masa en función de altura para algunos personajes", x = "altura", y = "masa")

ggplot(starwars) + 
    geom_point(aes(height, mass)) + 
    xlab("altura") + 
    ylab("masa")
\end{verbatim}

\hypertarget{ejes}{%
\subsection{Ejes}\label{ejes}}

Podemos manipular los ejes de diversas maneras, por ejemplo:

\begin{verbatim}
ggplot(starwars) + 
    geom_point(aes(height, mass)) + 
    scale_x_log10() + 
    scale_y_log10()

ggplot(starwars) + 
    geom_point(aes(height, mass)) + 
    xlim(c(0, 200)) + 
    ylim(c(0, 150))
\end{verbatim}

\hypertarget{podemos-usar-expresiones-luxf3gicas}{%
\subsection{Podemos usar expresiones
lógicas}\label{podemos-usar-expresiones-luxf3gicas}}

Podemos mapear experesiones hechas con operadores lógicos a elementos
estéticos. Los operadores lógicos generan una variable auxiliar
categórica.

\begin{verbatim}
# color de acuerdo a si son más o menos cuarentones
ggplot(starwars) + 
    geom_point(aes(height, mass, color=birth_year > 40))
# forma de acuerdo a si no nacieron en Tatooine
ggplot(starwars) + 
    geom_point(aes(height, mass, shape=homeworld != "Tatooine"))
# tamaño de acuerdo al año de nacimiento
ggplot(starwars) + 
    geom_point(aes(height, mass,  size=birth_year))
\end{verbatim}

\hypertarget{facets}{%
\subsection{Facets}\label{facets}}

Otra manera es dividir un gráfico en más de un gráfico de acuerdo a una
variable. Debemos usar una \texttt{formula}, pero que por ahora podemos
pensar como un tilde antes de una variable.

\begin{verbatim}
ggplot(starwars) + 
     geom_point(aes(height, mass)) + facet_grid(~gender)

# pasados de peso para los distintos géneros
ggplot(starwars) + 
    geom_point(aes(height, mass, color = mass/(height/100)^2 > 25)) + facet_grid(~gender) + labs (color = "BMI overweight")
\end{verbatim}

\hypertarget{otros-tipos-de-gruxe1ficos}{%
\subsection{Otros tipos de gráficos}\label{otros-tipos-de-gruxe1ficos}}

\begin{verbatim}
# histogramas
ggplot(starwars, aes(height)) + geom_histogram()
ggplot(starwars. aes(height)) + geom_histogram(bins = 50)
# densidades
ggplot(starwars, aes(height)) + geom_density()
ggplot(starwars, aes(height)) + geom_density(bw = 0.7)
# boxplots (para cada valor que quedó en gender, sobre la variable mass)
ggplot(starwars %>% filter(gender == "male" | gender == "female")) + geom_boxplot(aes(gender, mass))
\end{verbatim}

\end{document}
